\documentclass[11pt,a4paper]{report}
%\usepackage[italian]{babel}
\usepackage{graphicx}
\usepackage{wrapfig}
\usepackage{verbatim}
\usepackage{listings}
\usepackage{xcolor}
\usepackage{hyperref}
\usepackage{float}
\usepackage{titlesec}

\titleformat{\part}[display]
  {\Huge\bfseries}
  {}
  {0pt}
  {}

\titleformat{\chapter}[display]
 {\huge\bfseries}
 {}
 {0pt}
 {}

 \titleformat{\section}[display]
 {\Large\bfseries}
 {}
 {0pt}
 {\thesection\  \ }

\def\UrlBreaks{\do\/\do-}
\hypersetup{
    colorlinks=true,
    linkcolor=black,
    filecolor=magenta,
    urlcolor=cyan,
}
%\usepackage{url}
\definecolor{codegreen}{rgb}{0,0.6,0}
\definecolor{codegray}{rgb}{0.5,0.5,0.5}
\definecolor{codepurple}{rgb}{0.58,0,0.82}
\definecolor{backcolour}{rgb}{0.95,0.95,0.92}

\lstdefinestyle{mystyle}{
    backgroundcolor=\color{backcolour},
    commentstyle=\color{codegreen},
    keywordstyle=\color{magenta},
    numberstyle=\tiny\color{codegray},
    stringstyle=\color{codepurple},
    basicstyle=\ttfamily\footnotesize,
    breakatwhitespace=false,
    breaklines=true,
    captionpos=b,
    keepspaces=true,
    numbers=left,
    numbersep=5pt,
    showspaces=false,
    showstringspaces=false,
    showtabs=false,
    tabsize=2
}
\graphicspath{ {images/} }
\title{Orbital Mechanics Project}

\author{Marwan Alkady\\ Pedro Bossi N\'{u}\~{n}ez \\\ Demartini Davide\\ Iafrate Davide\\}
\date{\today}
%
\begin{document}
%%%%%%%%%%%%%%%%%%%%%%%%%%%%%%%%%%%%%%%%%%%%%%%%%%%%%%%%%%%%%%%%%%%
\begin{titlepage}
	\clearpage\thispagestyle{empty}
	\centering

%	Titles
%	Information about the University

   \centering \includegraphics[scale=0.7]{logo}

   \vspace{0.5cm}

	{\Huge\textbf{Politecnico di Milano} \\
Master of Science in\\ Space Engineering \\
		 \par}
		\vspace{3cm}
	{\Huge{Orbital Mechanics Project}} \\
	%\vspace{1cm}
	%{\large \textbf{xxxxx} \par}
	\vspace{4cm}
	{\LARGE Marwan Alkady\\ Pedro Bossi N\'{u}\~{n}ez \\ Davide Demartini\\ Davide Iafrate\\ \par}

%	Set the date
\vspace{1cm}
	{\Large A.Y. 2020-2021 \par}

	\pagebreak

\end{titlepage}
%%%%%%%%%%%%%%%%%%%%%%%%%%%%%%%%%%%%%%%%%%%%%%%%%%%%%%%%%%%%%%%%%%%%
%\frontmatter % First part of the book
\maketitle
%
\tableofcontents
%
%\mainmatter % Second part of the book

\part{Assignment 1: Interplanetary Explorer Mission}
\chapter{Mission requirements}
% table with:
% Departure Planet
% flyby planet
% arrival planet
% minimum departure and maximum arrival dates
\chapter{Mission analysis outputs}
\section{Design process}
\subsection{Initial choice for the time windows}
\subsection{Additional constraints considered}
\subsection{Transfer options exploration, analysis and comparison}
\subsection{Selection of the final solution}
% with plots and stuff
\section{Final solution}
\subsection{Heliocentric trajectory}
% Departure, flyby and arrival times.
% Plot of the heliocentric trajectory,together with the orbits of
% the three planets and their positions at departure, flyby and arrival.
\subsection{Powered gravity assist}
% Altitude of the closest approach.
% Time duration of the flyby (considering a finite SOI).
% Compare total flyby \Delta v with the cost of the manoeuvre \Delta v_{ga}
% Plot of the incoming and outcoming hyperbola arcs
\subsection{Cost of the mission}
%\Delta v_{dep}, \Delta v_{ga}, \Delta v_{arr}.


\part{Assignment 2: Planetary Explorer Mission}
\chapter{Mission requirements}
% table with:
% Central Planet
% Nominal operational orbit
% Orbit perturbations to be considered
% Ratio of satellite and Earth revolutions for the repeating ground track
\chapter{Mission analysis outputs}
\section{Nominal orbit}
%initial values and main characteristics
\section{Ground track}
% GT considering only J2 as perturbation and GT considering the assigned perturbation
\section{Orbit Propagation}
\subsection{In Cartesian coordinates}
\subsection{Keplerian elements through Gauss’ planetary equations}
\section{History of the Keplerian elements}
% plot the history of every keplerian element
\subsection{comparison of the propagation methods}
%in terms of accuracy, computational time etc.
\section{Orbit evolution representation}
\section{HF filtering}
%Maybe we can do it when we plot the history of the Keplerian elements
\section{Comparison with real data}
\subsection{Satellite selection}
\subsection{Comparison with our model}
\part{Bibliography}

%
%\bibliographystyle{plain}
%\bibliography{mybib}
%\printindex
\end{document}
